\documentclass[12pt,a4paper,twoside]{article}
\usepackage[utf8]{inputenc}
\usepackage[english]{babel}
\usepackage{amsmath}
\usepackage{amsfonts}
\usepackage{amssymb}
\usepackage{graphicx}
\usepackage[left=2cm,right=2cm,top=2cm,bottom=2cm]{geometry}

\usepackage{parcolumns}
\usepackage{siunitx}

\usepackage{pstricks}
\usepackage{pst-optexp}
\usepackage{xkeyval}

\author{Davide Bazzanella}
\title{Multidimensional Imaging of Ultrafast Laser Pulses}
%\institute{Imperial College London} 
\date{3$^{\mathrm{rd}}$ May 2016} 

\begin{document}
\pagenumbering{roman}	% numeration in roman numbers
\begin{titlepage}
\begin{center}

%\includegraphics[width=0.7\textwidth]{unitn_logo.png}~\\[1.2cm]

\hrule
$$$$
$$$$

\textsc{\LARGE \textbf{IMPERIAL COLLEGE LONDON}}\\[0.5cm]
\textsc{\LARGE \textbf{DEPARTMENT OF PHYSICS}}\\[1.3cm]
\textsc{\Large BSc PROJECT}\\[0.5cm]
\textsc{\Large Final Project Report}\\[1.3cm]

\vfill
\vfill
{ \huge \bfseries Multidimensional Imaging}\\
\vfill
{ \huge \bfseries of Ultrafast Laser Pulses}\\
\vfill
\vfill
\vfill

% WARNING! It need parcolumns package to work!
\begin{parcolumns}{2}
   \colchunk[1]{	
   				\Large Supervisor:\\
   				\LARGE \textbf{Tobias Witting}
   				}
   \colchunk[2]{
   				\Large Student:\\ \LARGE \textbf{Davide Bazzanella}
   				}
\end{parcolumns}
\vfill
\vfill
\vfill
\vfill
\vfill
\vfill
\vfill
\vfill
\vfill

\large ACADEMIC YEAR 2015/2016
\hrule
\vfill
\large TERM 2
\vfill
\vfill

\end{center}
\end{titlepage}
\cleardoublepage
\section*{Abstract}
We replicate a method for ultrafast laser pulses multidimensional characterisation.
The technique is based on fourier-transform interferometry and is spatially resolved in three dimensions.
The sample beam is interfered with a copy of itself delayed and passed through a pinhole, therefore spatially uniform.
The interferogram is acquired on each pixel of a CCD(??) camera by scanning the delay between the sample and the reference pulse and allow us to gather high-resolution spatially resolved information on the phase difference between the two pulses.
\subsubsection*{Keywords:} Laser metrology, femtosecond pulses, ultrafast lasers, multidimensional characterisation, fourier-transform interferometry
\cleardoublepage
\tableofcontents

\cleardoublepage
\pagenumbering{arabic}
\section{Introduction}
Modern mode-locked lasers are able to generate ultrafast laser pulses with a duration down to few femtoseconds\ref{ultrafast} and with high repetition rates.
Those pulses are too short to be characterised by any present-day detector, therefore spectroscopic and interferometric techniques are largely used in laser metrology.

** talk about those techniques **

Those techniques are usually implemented considering the laser pulse transversely homogeneous.
This is often not the case, because distortions in the pulse spatial distribution may be introduced even by the simplest optical element\ref{lenses}.

Mutidimensional characterisation of a ultrafast laser pulse is not simple: for example, the FROG and the SPIDER techniques can be scaled to allow one spatial dimension characterisation.

\subsection{Aims}
The aim of this project is to build a Mach-Zender interferometer to obtain a spatially resolved interferogram of the \SI{3.5}{\fs} laser pulse.
Then, by adding the phase information gathered by using the SEA-F-SPIDER, to characterise the beam spatial distribution in three dimensions

\section{Fourier transform interferometry}
\subsection{Autocorrelation interferometry}
Multidimensional characterisation of femtosecond laser pulses is difficult for a number of reasons. The main 
\subsection{Fourier transform interferometry}
A special case of the autocorrelation interferometry is the fourier-transform interferometry.
It is based on the information gathered by the autocorrelation interferometry.
\section{System setup}
\subsection{Laser Pulses Source}
**As a laser pulse source we are using a blah blah Ti:Sapphire blah blah 3.5 fs blah blah.**
** as prealignment beam we are using the one generated by a HeNe laser, producing a collimated beam at $\lambda_{HeNe} = \SI{632.8}{\nm}$ **
\clearpage
\subsection{Mach-Zender interferometer}
%\begin{center}
\begin{figure}
	\centering
	\begin{pspicture}(14,10)
		\pnodes(0,8){IN}(14,4){OUT}
		\pnodes(2,8){BSA}(7,4){BSB}
		\pnodes(2,2){C}(4,2){D}(4,4){E}(6,4){Focus}
		\pnodes(11,8){G}(11,6){H}(7,6){I}
		
		\addtopsstyle{OptComp}{mirrorwidth=1.5, mirrortype=extended}
		\addtopsstyle{OptComp}{bssize=1.5, bsstyle=plate}
		\addtopsstyle{Beam}{fillstyle=solid, fillcolor=red!50!white, linecolor=red, opacity=0.2}
		
		\beamsplitter(IN)(BSA)(C){BS1}
		
		\mirror(BSA)(C)(D){M1}
		\mirror(C)(D)(E){M2}
		\oapmirror[oapmirroraperture=1.5](D)(E)(Focus){OAP}
		\pinhole[outerheight=1,innerheight=0.1,phlinewidth=0.1](Focus)(Focus){PH}
		
		\beamsplitter(I)(BSB)(OUT){BS2}
		
		\mirror(BSA)(G)(H){M3}
		\mirror(G)(H)(I){M4}
		\mirror(H)(I)(BSB){M5}
		
		\drawwidebeam[beamwidth=0.4](IN){1-6}(OUT)
		\drawwidebeam[beamwidth=0.4](IN)(BSA){7-9}{6}(OUT)
		
		\optbox[optboxsize=4 4](8,7)(14,7){Z}
		\rput[r](12.5,7.15){$\tau$\psline[arrows=<->](-0.65, -0.15)(0.35, -0.15)}
	\end{pspicture}
	\label{fig-MZ}
	\caption{Mach-Zender interferometer}
\end{figure}
%\end{center}

The two arms of the interferometer have two mirrors to avoid retroreflection into the laser cavity.

\subsection{SEA-F-SPIDER}
\section{Data Processing}
\subsection{Diagram}
\section{Results}
\subsection{Simple case}
\section{Conclusions}
\subsection{Improvements}
\subsection{Summary}

\clearpage
\begin{thebibliography}{9}
\bibitem{miranda} Miranda, Miguel, et al. "Spatiotemporal characterization of ultrashort laser pulses using spatially resolved Fourier transform spectrometry." Optics letters 39.17 (2014): 5142-5145.
\bibitem{walmsley} Walmsley, Ian A., and Christophe Dorrer. "Characterization of ultrashort electromagnetic pulses." Advances in Optics and Photonics 1.2 (2009): 308-437.
\bibitem{witting} Witting, Tobias, et al. "Characterization of high-intensity sub-4-fs laser pulses using spatially encoded spectral shearing interferometry." Optics letters 36.9 (2011): 1680-1682.
\bibitem{trebino} Trebino, Rick, and Daniel J. Kane. "Using phase retrieval to measure the intensity and phase of ultrashort pulses: frequency-resolved optical gating." JOSA A 10.5 (1993): 1101-1111.


\end{thebibliography}
\end{document}