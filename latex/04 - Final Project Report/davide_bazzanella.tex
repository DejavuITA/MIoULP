\documentclass[12pt,a4paper,twoside]{article}
\usepackage[utf8]{inputenc}
\usepackage[english]{babel}
\usepackage{amsmath}
\usepackage{amsfonts}
\usepackage{amssymb}
\usepackage{graphicx}
\usepackage[left=2cm,right=2cm,top=2cm,bottom=2cm]{geometry}

\usepackage{parcolumns}
\usepackage{siunitx}

\usepackage{pstricks}
\usepackage{pst-optexp}
\usepackage{xkeyval}

\author{Davide Bazzanella}
\title{Multidimensional Imaging of Ultrafast Laser Pulses}
%\institute{Imperial College London} 
\date{3$^{\mathrm{rd}}$ May 2016} 

\begin{document}
\pagenumbering{roman}	% numeration in roman numbers
\begin{titlepage}
\begin{center}

%\includegraphics[width=0.7\textwidth]{unitn_logo.png}~\\[1.2cm]

\hrule
$$$$
$$$$

\textsc{\LARGE \textbf{IMPERIAL COLLEGE LONDON}}\\[0.5cm]
\textsc{\LARGE \textbf{DEPARTMENT OF PHYSICS}}\\[1.3cm]
\textsc{\Large BSc PROJECT}\\[0.5cm]
\textsc{\Large Final Project Report}\\[1.3cm]

\vfill
\vfill
{ \huge \bfseries Multidimensional Imaging}\\
\vfill
{ \huge \bfseries of Ultrafast Laser Pulses}\\
\vfill
\vfill
\vfill

% WARNING! It need parcolumns package to work!
\begin{parcolumns}{2}
   \colchunk[1]{	
   				\Large Supervisor:\\
   				\LARGE \textbf{Tobias Witting}
   				}
   \colchunk[2]{
   				\Large Student:\\ \LARGE \textbf{Davide Bazzanella}
   				}
\end{parcolumns}
\vfill
\vfill
\vfill
\vfill
\vfill
\vfill
\vfill
\vfill
\vfill

\large ACADEMIC YEAR 2015/2016
\hrule
\vfill
\large TERM 2
\vfill
\vfill

\end{center}
\end{titlepage}
\cleardoublepage
\section*{Abstract}
We replicate a method for ultrafast laser pulses multidimensional characterisation.
The technique is based on Fourier transform interferometry and is spatially resolved in three dimensions.
The sample beam is interfered with a copy of itself delayed and passed through a pinhole, therefore spatially uniform.
The interferogram is acquired on each pixel of a CCD(??) camera by scanning the delay between the sample and the reference pulse and allow us to gather high-resolution spatially resolved information on the phase difference between the two pulses.
\subsubsection*{Keywords:} Laser metrology, femtosecond pulses, ultrafast lasers, multidimensional characterisation, Fourier transform interferometry
\cleardoublepage
\tableofcontents

\cleardoublepage
\pagenumbering{arabic}
\section{Introduction}
Modern mode-locked lasers are able to generate ultrafast laser pulses with a duration down to few femtoseconds \cite{tamura93,schriever14,yu30} and with high repetition rates.
Those pulses are too short to be characterised by any present-day detector, therefore spectroscopic and interferometric techniques are largely used in laser metrology.

** talk about those techniques **

Frequency-Resolved Optical Gating (FROG) is a spectrographic method

Spectral-Phase Interferometry for Direct Electric-field Reconstruction (SPIDER) is an interferometric method

** talk about those techniques **

Those techniques are usually implemented considering the laser pulse transversely homogeneous.
This is often not the case, because distortions in the pulse spatial distribution may be introduced even by the simplest optical element \cite{bor}.

Mutidimensional characterisation of a ultrafast laser pulse is not simple: for example, the FROG and the SPIDER techniques can be scaled to allow one spatial dimension characterisation.\\
\textit{Being able to obtain the 3D distribution of a laser pulse with those techniques is not as simple.}

\subsection{Aims}
The aim of this project is to spatially and temporally characterise an ultrafast laser pulse of \SI{3.5}{\fs} duration, generated by a Ti:Sapphire laser.

Our plan is to build a Mach-Zender interferometer to obtain a spatially resolved interferogram of the laser pulse, and convert it to the frequency domain via Fourier transform.
Then, by adding information on the frequency dependance of the phase, gathered using the SEA-F-SPIDER, to characterise the beam spatial distribution in three dimensions.

\section{Fourier transform interferometry}
Among the different techniques, we have chosen to develop a spatially resolved version of the Fourier transform interferometry.
This method is based on the interference between the test beam and a small homogeneous portion of a copy of itself
\subsection{Autocorrelation interferometry}
Consider an unknown laser pulse, its time complex field and the corresponding field in the spectral domain are given by
\begin{gather}
	U(t) = |U(t)|e^{i\phi(t)} \\
	\tilde{U}(\omega) = |\tilde{U}(\omega)|e^{i\psi(\omega)}
	\label{eq_def}
\end{gather}
The camera measure the intensity of the pulses over a time much longer than a single cycle, therefore we may write that for each delay $\tau$:
\begin{align}
	I(\tau) 	&= \int|U(t)+U_r(t-\tau)|^2\mathrm{d}t \\
			&= \int|U(t)|^2\mathrm{d}t + \int|U_r(t)|^2\mathrm{d}t\\
			&\quad + \int U(t)U_r^*(t-\tau)\mathrm{d}t + \int U^*(t)U_r(t-\tau)\mathrm{d}t 
	\label{eq_autocorr}
\end{align}
As we can see the first two terms are actually just the irradiance of the test beam $I_0 = \int|U(t)|^2\mathrm{d}t$ and of the reference beam $I_r = \int|U_r(t)|^2\mathrm{d}t$ and are therefore constant for each value of delay $\tau$.

Those component may be isolated and subtracted to the interferogram either by separate measurement or by removing the peak at $\omega = 0$ in the frequency domain.
\subsection{Fourier transform interferometry}
A special case of the autocorrelation interferometry is the Fourier transform interferometry.
It is based on the information gathered by the autocorrelation interferometry.

By applying the Fourier transform to Eq. (\ref{eq_autocorr}) we obtain
\begin{equation}
	\mathrm{FT}\lbrace I(\tau)\rbrace =
	\mathrm{FT}\lbrace\int|U(t)|^2\mathrm{d}t + \int|U_r(t)|^2\mathrm{d}t\rbrace
	+ \tilde{U}(\omega)\tilde{U}_r^*(\omega)
	+ \tilde{U}^*(-\omega)U_r(-\omega)
\end{equation}
where $\tilde{U}(\omega)$ is the spectral domain complex field, as in Eq. (\ref{eq_def}).

The first term is the Fourier transform of a constant and is represented in the frequency domain by a delta at $\omega = 0$.
The second and third terms are represented by two peaks around $\omega_0$ and $-\omega_0$, where $\omega_0$ the main frequency of the pulse.
Considering only the second term, which can be isolated from the term at zero and at $-\omega$, it may be rewritten as:
\begin{equation}
	A(\omega)=|\tilde{U}(\omega)||\tilde{U}_r^*(\omega)|e^{i[\psi(\omega)-\psi_r(\omega)]}
\end{equation}
and finally, we get
\begin{equation}
	\tilde{U}(\omega) = |\tilde{U}(\omega)|e^{i\psi(\omega)} = \frac{A(\omega)}{|\tilde{U}_r^*(\omega)|}e^{i\psi_r(\omega)}
\end{equation}

\subsubsection{Multidimensional approach}
\begin{equation}
	\tilde{U}(x,y,\omega) = \frac{A(x,y,\omega)}{|\tilde{U}_r^*(x,y,\omega)|}e^{i\psi_r(x,y,\omega)}
\end{equation}
assumptions:
\begin{itemize}
\item the intensity of the reference beam is homogeneous in the transverse directions $|\tilde{U}(x,y,\omega)|=|\tilde{U}(x_0,y_0,\omega)|=|\tilde{U}(\omega)|$ where ($x_0$,$y_0$) are the coordinates of the center of the beam.
\item the phase is separable is a spatial component and an angular frequency component:\\ $\psi_r(x,y,\omega) = \psi_{r,sp}(x,y)\psi_{r,\omega}(\omega)$
\end{itemize}
Then:
\begin{equation}
	\tilde{U}(x,y,\omega) \approx \frac{A(x,y,\omega)}{|\tilde{U}_r^*(\omega)|}e^{i\psi_r(x,y,\omega)}
\end{equation}

\section{System setup}
\subsection{Laser Pulses Source}
**As a laser pulse source we are using a blah blah Ti:Sapphire blah blah 3.5 fs blah blah.**
** as prealignment beam we are using the one generated by a HeNe laser, producing a collimated beam at $\lambda_{HeNe} = \SI{632.8}{\nm}$ **
\clearpage
\subsection{Mach-Zender interferometer}
\begin{figure}[ht]
	\centering
	\begin{pspicture}(14,10)
		\pnodes(0,8){IN}(14,4){OUT}
		\pnodes(2,8){BSA}(7,4){BSB}
		\pnodes(2,2){C}(4,2){D}(4,4){E}(6,4){Focus}
		\pnodes(11,8){G}(11,6){H}(7,6){I}
		
		\addtopsstyle{OptComp}{mirrorwidth=1.5, mirrortype=extended}
		\addtopsstyle{OptComp}{bssize=1.5, bsstyle=plate}
		\addtopsstyle{Beam}{fillstyle=solid, fillcolor=red!50!white, linecolor=red, opacity=0.2}
		
		\beamsplitter(IN)(BSA)(C){BS1}
		
		\mirror(BSA)(C)(D){M1}
		\mirror(C)(D)(E){M2}
		\oapmirror[oapmirroraperture=1.5](D)(E)(Focus){OAP}
		\pinhole[outerheight=1,innerheight=0.1,phlinewidth=0.1](Focus)(Focus){PH}
		
		\beamsplitter(I)(BSB)(OUT){BS2}
		
		\mirror(BSA)(G)(H){M3}
		\mirror(G)(H)(I){M4}
		\mirror(H)(I)(BSB){M5}
		
		\drawwidebeam[beamwidth=0.4](IN){1-6}(OUT)
		\drawwidebeam[beamwidth=0.4](IN)(BSA){7-9}{6}(OUT)
		
		\optbox[optboxsize=4 4](8,7)(14,7){Z}
		\rput[r](12.5,7.15){$\tau$\psline[arrows=<->](-0.65, -0.15)(0.35, -0.15)}
	\end{pspicture}
	\label{fig-MZ}
	\caption{Mach-Zender interferometer: BS1 and BS2 are the two 50/50 beamsplitter, M1-5 are plain mirrors, PH is the \SI{20}{\um} pinhole and OAP is the off-axis parabolic gold-plated mirror.}
\end{figure}

The two arms of the interferometer have two mirrors to avoid retroreflection into the laser cavity.

We decided to keep the reference arm fixed and build the moving platform in the test arm.

\subsection{SEA-F-SPIDER}
\section{Data Processing}
\subsection{Diagram}
\section{Results}
\subsection{Simple case}
\section{Conclusions}
\subsection{Improvements}
As a technical improvement corner cube rectroreflector should be used for a better alignment and less sensibility to the mirrors movement.
\subsection{Summary}

\clearpage
\begin{thebibliography}{99}
\bibitem{tamura93} Tamura, K., et al. "77-fs pulse generation from a stretched-pulse mode-locked all-fiber ring laser." Optics letters 18.13 (1993): 1080-1082.
\bibitem{schriever14} Schriever, Christian, et al. "Tunable pulses from below 300 to 970 nm with durations down to 14 fs based on a 2 MHz ytterbium-doped fiber system." Optics letters 33.2 (2008): 192-194.
\bibitem{yu30} Yu, Tae Jun, et al. "Generation of high-contrast, 30 fs, 1.5 PW laser pulses from chirped-pulse amplification Ti: sapphire laser." Optics express 20.10 (2012): 10807-10815.
\bibitem{bor} Bor, Zsolt. "Distortion of femtosecond laser pulses in lenses." Optics letters 14.2 (1989): 119-121.
\bibitem{miranda} Miranda, Miguel, et al. "Spatiotemporal characterization of ultrashort laser pulses using spatially resolved Fourier transform spectrometry." Optics letters 39.17 (2014): 5142-5145.
\bibitem{walmsley} Walmsley, Ian A., and Christophe Dorrer. "Characterization of ultrashort electromagnetic pulses." Advances in Optics and Photonics 1.2 (2009): 308-437.
\bibitem{witting} Witting, Tobias, et al. "Characterization of high-intensity sub-4-fs laser pulses using spatially encoded spectral shearing interferometry." Optics letters 36.9 (2011): 1680-1682.
\bibitem{trebino} Trebino, Rick, and Daniel J. Kane. "Using phase retrieval to measure the intensity and phase of ultrashort pulses: frequency-resolved optical gating." JOSA A 10.5 (1993): 1101-1111.


\end{thebibliography}
\end{document}