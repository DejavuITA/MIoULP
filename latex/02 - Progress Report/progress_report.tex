\documentclass[12pt,a4paper]{report}
\usepackage[utf8]{inputenc}
\usepackage[english]{babel}
\usepackage{amsmath}
\usepackage{amsfonts}
\usepackage{amssymb}
\usepackage{graphicx}
\usepackage[left=2cm,right=2cm,top=2cm,bottom=2cm]{geometry}

\usepackage{siunitx}

\usepackage{fancyhdr} %header

\pagestyle{fancyplain}

\setlength{\headheight}{27.2pt}
\fancyhead[R]{16 February 2016}
\fancyhead[L]{Multidimensional Imaging of Ultrafast Laser}
%\setlength{\footskip}{40pt}
%\fancyfoot[R]{3rd Year Project}
\fancyfoot[L]{Davide Bazzanella, CID 01149516}
%\fancyfoot[C]{a\\\thepage}

\author{Davide Bazzanella}
\title{Project Plan}


\begin{document}
\textbf{\Huge{Progress Report}}
\section*{Achievements and Difficulties}
At the beginning, I had to become familiar with the environment and the equipement in the laboratory, but this was just for a short period of time.
\subsubsection*{Build a Mach-Zender interferometer with a moving arm}
The first step of the project was to build a Mach-Zender interferometer.
In doing so, we have encountered a few difficulties, originated mainly by one factor, which is the need for the two output beams to be collimated.

The first problem was that the movement of the arm should not affect the path in other way than its length, because even a little offset in the beam direction may led to a movement of the output beam.
We tried to limit this phenomenon at our best.
A possible upgrade would be to use retroreflectors (corner cube retroreflector, spherical retroreflector).

Another difficulty was to keep the beam in the same horizontal plane, which is not required, but leads to a better alignment in the output.

\subsubsection*{Test the interferometer with HeNe laser}
The second step was to test the interferometer with light coming from a HeNe laser generating a continuous wave at $\lambda = \SI{633}{\nm}$.

It is important to have the right input direction, otherwise it could affect the alignment in the output.
Hence we used a two mirror stage to achieve the best alignement possible for the input beam.

The requirement for interference in a Mach-Zender interferometer is that the difference in length between the two arms is within the coherence length of the light.
The HeNe beam has a high coherence length, therefore it was useful to accomplish a first calibration of the length of the two arms with this laser.

\subsubsection*{Test the interferometer with white low coherence light}
The next step was to test the interferometer with white (broadband) low coherence light, to calibrate the two arms to less than \SI{1}{\um} in length difference.

Using a two mirror stage for the white light input beam, we obtained alignement of the two output beams.
Then we proceeded in moving one arm to calibrate the interferometer, but we did not see any interference pattern.
Hence we stepped back and tried to recalibrate the interferometer with the HeNe laser.
We checked length of the arms, beams directionality, beams alignment and we replaced some components (motor of moving platform).

Even though the interferometer was calibrated in the best way with the HeNe laser, we were unable to obtain interference with the white light.

Therefore we agreed to go to the next step anyway.

\subsubsection*{Replacement with parabolic mirror and pinhole}
The last step we completed was the replacement of the third mirror in the reference arm with a parabolic mirror and a pinhole, thus creating a reference beam.
We decided to replace the third mirror in the fixed arm to avoid alignement aberrations due to the arm movement and the parabolic mirror.


\section*{Work in progress}
\subsubsection*{Recalibration}
This week we are in the progress of calibrating the setup with the parabolic mirror.
With this system, we obtained both interference with the HeNe laser and the Ti:Sapphire laser.

\section*{Work Remaining}
\begin{enumerate}
\item Use the HeNe laser to finish adjusting the setup with the cheap camera as sensor. (\textit{system setup})
\item Use the HeNe laser, with its known shape (Gaussian Beam), to develop the code to obtain the spatio-temporal imaging of the beam from the sensor output. (\textit{data collection and analysis})
\item Use the Ti:Sapphire laser to adjust the setup with a high quality camera. (\textit{system setup})
\item Characterize the \SI{3.5}{\fs} ultrashort laser pulse of the Ti:Sapphire laser with the interferometer built and the algorithms developed so far. (\textit{data collection and analysis})
\end{enumerate}
Note: the second point may be carried forward before, after or at the same time of the third point, depending on the availability of the Ti:Sapphire laser.

\end{document}