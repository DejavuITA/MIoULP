\documentclass[12pt,a4paper]{report}
\usepackage[utf8]{inputenc}
\usepackage[english]{babel}
\usepackage{amsmath}
\usepackage{amsfonts}
\usepackage{amssymb}
\usepackage{graphicx}
\usepackage[left=2cm,right=2cm,top=2cm,bottom=2cm]{geometry}

\usepackage{siunitx}

\usepackage{fancyhdr} %header

\pagestyle{fancyplain}
\setlength{\headheight}{27.2pt}
\fancyhead[R]{3rd Year Project\\Multidimensional Imaging of Ultrafast Laser Pulses}
\fancyhead[L]{Davide Bazzanella,\\CID 01149516}

\author{Davide Bazzanella}
\title{Project Plan}


\begin{document}
\textbf{\Huge{Project Plan}}
\section*{Introduction and goals}
The goal of this project is to develop a system capable to obtain a multidimensional image of an ultrashort laser pulse.

The last term, two people worked at this project and developed a Push-Broom system and a CASSI system.
The first one worked, while the latter didn't, probably because of the difficult implementation of the algorithms.

During this term, I will work side by side with one of the two students who worked in the last term.
Our aim is to develop a Mach-Zender interferometer to measure the spatio-temporal profile of an ultrashort (\SI{3.5}{\fs}) laser pulse from a Ti:Sapphire laser, by employing Fourier-transform spectroscopy techniques.

\section*{Project plan}
\begin{enumerate}
\item Build a Mach-Zender interferometer with a moving arm
%	\begin{itemize}
%	\item the beams must be in the same horizontal plane.
%	\item the beam must be aligned to the arm's movement.
%	\item the output of the beams must be collimated.
%	\end{itemize}
\item Test the interferometer with HeNe laser ($\lambda = \SI{633}{\nm}$)
\item Test the interferometer with white low coherence light, to calibrate the two arms to under \SI{1}{\um} of lenght difference.
\item Use the interferometer to set up the sensor (camera).
\item Develop the code to obtain the spatio-temporal imaging of the beam from the sensor output.
\item Replace the third mirror in one arm with a parabolic mirror and a pinhole, thus creating a reference beam.
\item Recalibrate the interferometer.
\item Use the interferometer to characterize the \SI{3.5}{\fs} ultrashort laser pulse of the Ti:Sapphire laser.
\end{enumerate}

\section*{Reading material}
\begin{itemize}
\item Walmsley, Ian A., and Christophe Dorrer. "Characterization of ultrashort electromagnetic pulses." Advances in Optics and Photonics 1.2 (2009): 308-437.
\item Witting, Tobias, et al. "Characterization of high-intensity sub-4-fs laser pulses using spatially encoded spectral shearing interferometry." Optics letters 36.9 (2011): 1680-1682.
\item Miranda, Miguel, et al. "Spatiotemporal characterization of ultrashort laser pulses using spatially resolved Fourier transform spectrometry." Optics letters 39.17 (2014): 5142-5145.
\item Trebino, Rick, and Daniel J. Kane. "Using phase retrieval to measure the intensity and phase of ultrashort pulses: frequency-resolved optical gating." JOSA A 10.5 (1993): 1101-1111.
\end{itemize}

%\begin{thebibliography}{9}
%\bibitem{miranda} Miranda, Miguel, et al. "Spatiotemporal characterization of ultrashort laser pulses using spatially resolved Fourier transform spectrometry." Optics letters 39.17 (2014): 5142-5145.
%\end{thebibliography}

\end{document}